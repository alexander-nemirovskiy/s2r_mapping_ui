\documentclass{scrreprt}
\usepackage{listings}
\usepackage{underscore}
\usepackage{graphicx}
\usepackage[bookmarks=true]{hyperref}
\usepackage[utf8]{inputenc}
\usepackage[english]{babel}
\hypersetup{
	bookmarks=false,    % show bookmarks bar?
	pdftitle={Software Requirement Specification},    % title
	pdfauthor={Alexander Nemirovskiy},                     % author
	pdfsubject={TeX and LaTeX},                        % subject of the document
	pdfkeywords={TeX, LaTeX, graphics, images}, % list of keywords
	colorlinks=true,       % false: boxed links; true: colored links
	linkcolor=blue,       % color of internal links
	citecolor=black,       % color of links to bibliography
	filecolor=black,        % color of file links
	urlcolor=purple,        % color of external links
	linktoc=page            % only page is linked
}%

\date{}

\usepackage{hyperref}

\begin{document}
	
\newcommand{\version}[0]{0.2.1}
\newcommand{\proxy}[0]{\texttt{proxy.conf}}
\newcommand{\jaxb}[0]{\texttt{jaxb_impl-0.1.3.jar}}
\newcommand{\model}[0]{\texttt{model.bin}}

\begin{flushright}
	\rule{16cm}{5pt}\vskip1cm
	\begin{bfseries}
		\Huge{SOFTWARE REQUIREMENTS\\ SPECIFICATION}\\
		\vspace{1.5cm}
		for\\
		\vspace{1.5cm}
		S.M.A.R.T\\
		\large{Sprint Mapping and Annotation Recommendation Tool\\}
		\vspace{1.5cm}
		\LARGE{Version \version}\\
		\vspace{5cm}
		Prepared by : Alexander Nemirovskiy\\
		\today\\
	\end{bfseries}
\end{flushright}

\pagebreak
%\chapter{Requirements}
\section[1]{Information}
\paragraph{}The current software version \version \space is a web application that consists of two docker containers activated through a docker-compose file provided with the installation package. 

A backend API container is created exposing a process running on the host machine and listening on port 8081 by default. This process represents the core of the application. It is not necessary to expose the port it is using as it is meant for internal communication only unless the goal is to create alternative UI to what is provided in the package. The application is designed to work as a standalone API as well.

An additional NginX container is orchestrating incoming http requests and acting as a reverse proxy towards the internal API process. It expects port 80 to be exposed on the host machine by default. It is possible to change this port if necessary as an \texttt{.env} file is provided in the installation package. 

This container is secondary to the application usage and it is meant only to facilitate the installation process and can be replaced with a custom service that acts as a web server for the application's UI.



\section{Technical requirements}
\paragraph{}The application was tested on the following configurations:
\begin{itemize}
	\item CentOS/RHEL based Amazon Linux AMI 2018.03 x64
	\item 8 vCPUs 2.3 GHz, Intel Broadwell E5-2686v4
	\item 32 GB RAM
	\item 32 GB internal dick storage
	\item docker and docker-compose installed
	\item npm v6.14.x and angular CLI v9.1.x installed 
\end{itemize}
Although not necessary to replicate, these configurations allowed the application to be run asynchronously and in a multi-threaded environment which inherently augmented the throughput.


\paragraph{}The minimal set of requirements are defined as follows:
\begin{itemize}
	\item An operative system capable of running docker and compose version 3 or above
	\item a multi-core CPU with at least 2GHz  frequency
	\item at least 5GB of available storage on the disk
	\item at least 8GB of RAM
\end{itemize}


%\chapter{Configuration}
\section{Host machine setup}
\paragraph{}The application is delivered in a package containing the necessary modules to build and run the program. It expects the host machine to have at least docker installed and available on it. The default installation process requires some ports to be open as well for either internal usage or external calls to the program.

This package contains the following items:
\begin{itemize}
	\item an \texttt{app} folder containing the source code of the python backend.
	\item a \texttt{frontend/angularUI} folder containing the UI project to be compiled, built and later exposed through a web server.
	\item a \texttt{Dockerfile} and a \texttt{docker-compose} files to build and start the application.
	\item three folders, namely \texttt{uploads, input, output} that contain, or will contain, all the additional files used or generated by the application such as a model binary file and a jar file used by the backend services.
\end{itemize}
Do not change the contents of this folder before building the application, unless specified otherwise.

\section{Configuring the application}
\paragraph{Standard configuration} This is the default configuration option. It will be using ports 80 and 8081 on the host machine to run the application. These settings can be changed at any time according to the host machine availability and setup. The following steps describe what needs to be done in order to launch the application without any change to the settings.
\begin{enumerate}
	\item Extract the contents of the provided package into a folder. From now on this will be referred as the root folder of the application.
	\item put the \texttt{model.bin} file into the  folder that should already contain a \texttt{jar}: the application will be using both these files during the execution of the mapping process.
	\item Navigate back to the root directory and start the docker-compose process using the \texttt{docker-compose.yml}. This will start the build process of the application for both frontend and backend parts.	
	\item (This step is optional if the package already contains the mentioned files. If those files have been downloaded separately this step is mandatory) Once the building process is finished put the \jaxb \space and the \model \space files into the \texttt{input} folder. If named otherwise, please rename the external model file to \model \space and do the same for \jaxb.
	\item Now start the docker containers using docker-compose once more.
\end{enumerate}

\paragraph{Additional information}
If it is required to change the internal API service port, both the Docker file in the root folder and the nginx proxy configuration file need to be modified according to the desired configuration: change the port number in the command line of the docker file that starts the uvicorn server for the API (it should be the last line). If the NginX service is used as well, the aforementioned change shall be reflected in the \texttt{upstream} block of the \proxy \space file. Both files shall contain the same port number for the application to work unless another custom service is to be used instead of NginX otherwise the API won't work. To summarize the needed steps:
\begin{enumerate}
	\item In the root folder, open the docker file and change the last parameter in line \texttt{RUN [$\dots$]} that indicates the port on which the API will be running
	\item Navigate to the \texttt{frontend/angularUI/service\_conf} folder and open the 
	
	\proxy \space file with a standard text editor.
	\item The setting inside \texttt{upstream} block represents the internal process port the backend API will be running on and to which the NginX will forward http requests. This is not meant to be exposed to an external user necessarily and it is for internal use only. Change the parameters to be identical to the choice made in the previous step.
\end{enumerate}
\end{document}